%===================================================================
\documentclass[twoside,a4paper,leqno]{article}
%===================================================================
\usepackage{mathpazo} % I like it.
\usepackage[bubblesheet]{mcq} 
\englishinfo %% this if you want English sentences.
\usepackage{polyglossia}
\setdefaultlanguage{english} %% it needs to be *after* mcq.

%===================================================================
\headline{Multiple Choice questions on Logic (2013-12-31)}
%% this is the "name" of the exam
%  http://www.math.ucla.edu/~tao/java/MultipleChoice/logic.txt
%  http://www.math.ucla.edu/~tao/java/MultipleChoice/MultipleChoice.html 

\puntigiusta{6} % points for a correct answer
\puntisbagliata{-1} % points for a wrong answer (when not explicitely stated)
\puntiempty{0} % points for non-response. 


%===================================================================
% The following is necessary only if bubblesheet...
% \UIDdigits{5}
% change it if the default value (6) does not fit.


%===================================================================
% This is the text appearing on the right of the UID matrix.
% It is safe to change it (but it should be not too long...)
%\renewcommand{\geninfo}{%
%Instructions: fill \textbf{completely} the bubbles
%with the digits of the SID (one for each column);
%in the lower part of the sheet, fill \textbf{completely}
%the bubble with the correct answers to the corresponding question.
%Use a black or dark blue pen or pencil,
%trying to fill completely the inside of the bubble.
%Write only in the designated areas.
%}
%\renewcommand{\ansinfo}{\textbf{Mark the answers of the multiple-choice questions}}
%\renewcommand{\uidname}{Student ID}
%\renewcommand{\Cognomename}{Last Name}
%\renewcommand{\Nomename}{First name}
%\renewcommand{\Firmaname}{Signature}

% the previous commands are equivalent to the command
% \englishinfo


% It is safe to renew \squarebox: empty for no-squarebox,
% or whatever Unicode character or symbol. It is the symbol appearing
% on the left of each possible item-answer. Check that the characters
% is included in the font...
% \renewcommand{\squarebox}{--}

% If changing the font, make sure that it's not too ugly.
% \setmainfont[mapping=tex-text]{Linux Libertine}
%===================================================================
% just some newcommands
\newcommand{\RR}{\mathbb{R}}
\newcommand{\CC}{\mathbb{C}}
\newcommand{\ZZ}{\mathbb{Z}}
\newcommand{\NN}{\mathbb{N}}
\renewcommand{\AA}{\mathbb{A}}
\newcommand{\EE}{\mathbb{E}}
\newcommand{\QQ}{\mathbb{Q}}
\newcommand{\PP}{\mathbb{P}}
\newcommand{\FF}{\mathbb{F}}
\newcommand{\KK}{\mathbb{K}}
\newcommand{\from}{\colon}
\newcommand{\vv}{\boldsymbol{v}}
\newcommand{\vw}{\boldsymbol{w}}

%===================================================================
\begin{document}
\bubblesheet[2]{14}{7} 
% [number of columns] {number of questions} {number of answers}

\begin{esercizi*}{}
% enviroment including the questions. 
% The second argument is the tile of the exercises section.

\begin{exerm}
  Let $X$ and $Y$ be statements.  If we know that $X$ implies $Y$, then we can also conclude that
\begin{rispm}[2] % the optional parameters is the number of columns
% default is 1 
\risp $X$ is true, and $Y$ is also true.
\fb{%
This is a feedback comment (optional). Useful only in the conversion
to HTML or for dynamical formats on MOODLE. 
}
\risp[-2]  % in square bracket this answer will get -2 points, instead of default.
$Y$ cannot be false.
\fb{This was terribly wrong.}
\risp[3] If $Y$ is true, then $X$ is true.
\risp[=] 
% This is the correct answer. 
If $Y$ is false, then $X$ is false.
\risp If $X$ is false, then $Y$ is false.
\risp $X$ cannot be false.
\risp[3.1415]
% you can set individually the points achieved by the answer with signed 
% integers of floats. 
 At least one of $X$ and $Y$ is true.
\end{rispm}


\end{exerm}
\begin{exerm}
  Let $X$ and $Y$ be statements.  If we want to DISPROVE the claim that "Both $X$ and $Y$ are true", we need to show that
\begin{rispm}
\risp[=]  At least one of $X$ and $Y$ are false.
\risp $X$ and $Y$ are both false.
\risp[0.5] $X$ is false.
\fb{This will indeed disprove "Both $X$ and $Y$ are true", but $X$ does not need to be false in order to disprove the above statement.}
\risp[0.5] $Y$ is false.
\fb{This will indeed disprove "Both $X$ and $Y$ are true", but $Y$ does not need to be false in order to disprove the above statement.}
\risp $X$ does not imply $Y$, and $Y$ does not imply $X$.
\risp Exactly one of $X$ and $Y$ are false.
\risp $X$ is true if and only if $Y$ is false.
\end{rispm}
\end{exerm}

\begin{exerm}
  Let $X$ and $Y$ be statements.  If we want to DISPROVE the claim that "At least one of $X$ and $Y$ are true", we need to show that
\begin{rispm}
\risp  At least one of $X$ and $Y$ are false.
\risp[=] $X$ and $Y$ are both false.
\risp $X$ is false.
\risp $Y$ is false.
\risp $X$ does not imply $Y$, and $Y$ does not imply $X$.
\risp Exactly one of $X$ and $Y$ are false.
\risp $X$ is true if and only if $Y$ is false.
\end{rispm}
\end{exerm}

\begin{exerm}
  Let $X$ and $Y$ be statements.  If we want to DISPROVE the claim that "$X \implies Y$", we need to show that
\begin{rispm}
\risp  $Y$ is true, but $X$ is false.
\risp[=] $X$ is true, but $Y$ is false.
\risp $X$ is false.
\risp $Y$ is false.
\risp $X$ and $Y$ are both false.
\risp Exactly one of $X$ and $Y$ are false.
\risp At least one of $X$ and $Y$ is false.
\end{rispm}
\end{exerm}

\begin{exerm}
  Let $P(x)$ be a property about some object $x$ of type $X$.  If we want to DISPROVE the claim that "$P(x)$ is true for all $x$ of type $X$", then we have to
\begin{rispm}
\risp[=]  Show that there exists an $x$ of type $X$ for which $P(x)$ is false.
\risp  Show that there exists an $x$ which is not of type $X$, but for which $P(x)$ is still true.
\risp  Show that for every $x$ in $X$, $P(x)$ is false.
\risp  Show that $P(x)$ being true does not necessarily imply that $x$ is of type $X$.
\risp  Assume there exists an $x$ of type $X$ for which $P(x)$ is true, and derive a contradiction.
\risp  Show that there are no objects $x$ of type $X$.
\fb{Actually, if there are no objects of type $X$, then the statement "$P(x)$ is true for all $x$ of type $X$" is automatically true (but vacuously so)!}
\risp  Show that for every $x$ in $X$, there is a $y$ not equal to $x$ for which $P(y)$ is true.
\end{rispm}
\end{exerm}

\begin{exerm}
  Let $P(x)$ be a property about some object $x$ of type $X$.  If we want to DISPROVE the claim that "$P(x)$ is true for some $x$ of type $X$", then we have to
\begin{rispm}
\risp  Show that there exists an $x$ of type $X$ for which $P(x)$ is false.
\risp  Show that there exists an $x$ which is not of type $X$, but for which $P(x)$ is still true.
\risp[=]  Show that for every $x$ in $X$, $P(x)$ is false.
\risp  Show that $P(x)$ being true does not necessarily imply that $x$ is of type $X$.
\risp  Assume that $P(x)$ is true for every $x$ in $X$, and derive a contradiction.
\risp[3.5]  Show that there are no objects $x$ of type $X$.
\fb{This will certainly disprove the claim, however, one does not always need X to be empty in order to disprove the claim.}
\risp  Show that for every $x$ in $X$, there is a $y$ not equal to $x$ for which $P(y)$ is true.
\end{rispm}
\end{exerm}

\begin{exerm}
  Let $P(n,m)$ be a property about two integers $n$ and $m$.  If we want to prove that "For every integer $n$, there exists an integer $m$ such that $P(n,m)$ is true", then we should do the following:
\begin{rispm}
\risp[=]  Let $n$ be an arbitrary integer.  Then find an integer $m$ (possibly depending on $n$) such that $P(n,m)$ is true.
\risp  Let $n$ and $m$ be arbitrary integers.  Then show that $P(n,m)$ is true.
\fb{This will definitely prove what we want, but is far too strong, it proves much more than what we need!}
\risp  Find an integer $n$ and an integer $m$ such that $P(n,m)$ is true.
\risp  Let $m$ be an arbitrary integer.  Then find an integer $n$ (possibly depending on $m$) such that $P(n,m)$ is true.
\risp  Find an integer $n$ such that $P(n,m)$ is true for every integer $m$.
\risp  Find an integer $m$ such that $P(n,m)$ is true for every integer $n$.
\fb{ This will prove what we want, it is too strong - it proves more than we need.}
\risp  Show that whenever $P(n,m)$ is true, then $n$ and $m$ are integers.
\end{rispm}
\end{exerm}

\begin{exerm}
  Let $P(n,m)$ be a property about two integers $n$ and $m$.  If we want to DISPROVE the claim that "For every integer $n$, there exists an integer $m$ such that $P(n,m)$ is true", then we need to prove that
\begin{rispm}
\risp[=]  There exists an integer $n$ such that $P(n,m)$ is false for all integers $m$.
\risp  There exists integers $n$,$m$ such that $P(n,m)$ is false.
\risp  For every integer $n$, and every integer $m$, the property $P(n,m)$ is false.
\risp  For every integer $n$, there exists an integer $m$ such that $P(n,m)$ is false.
\risp  For every integer $m$, there exists an integer $n$ such that $P(n,m)$ is false.
\risp  There exists an integer $m$ such that $P(n,m)$ is false for all integers $n$.
\risp  If $P(n,m)$ is true, then $n$ and $m$ are not integers.
\end{rispm}
\end{exerm}

\begin{exerm}
  Let $P(n,m)$ be a property about two integers $n$ and $m$.  If we want to DISPROVE the claim that "There exists an integer $n$ such that $P(n,m)$ is true for all integers $m$", then we need to prove that
\begin{rispm}
\risp  There exists an integer $n$ such that $P(n,m)$ is false for all integers $m$.
\risp  There exists integers $n$,$m$ such that $P(n,m)$ is false.
\risp  For every integer $n$, and every integer $m$, the property $P(n,m)$ is false.
\risp[=]  For every integer $n$, there exists an integer $m$ such that $P(n,m)$ is false.
\risp  For every integer $m$, there exists an integer $n$ such that $P(n,m)$ is false.
\risp  There exists an integer $m$ such that $P(n,m)$ is false for all integers $n$.
\risp  If $P(n,m)$ is true, then $n$ and $m$ are not integers.
\end{rispm}
\end{exerm}

\begin{exerm}
  Let $X$ and $Y$ be statements.  Which of the following strategies is NOT a valid way to show that "$X \implies Y$"?
\begin{rispm}
\risp  Assume that $X$ is true, and then use this to show that $Y$ is true.
\risp  Assume that $Y$ is false, and then use this to show that $X$ is false.
\risp  Show that either $X$ is false, or $Y$ is true, or both.
\risp  Assume that $X$ is true, and $Y$ is false, and deduce a contradiction.
\risp[=]  Assume that $X$ is false, and $Y$ is true, and deduce a contradiction.
\risp  Show that $X$ implies some intermediate statement $Z$, and then show that $Z \implies Y$.
\risp  Show that some intermediate statement $Z \implies Y$, and then show that $X \implies Z$.
\end{rispm}
\end{exerm}

\begin{exerm}
  Suppose one wishes to prove that "if all $X$ are $Y$, then all $Z$ are $W$".  To do this, it would suffice to show that
\begin{rispm}
\risp[=]  All $Z$ are $X$, and all $Y$ are $W$.
\risp  All $X$ are $Z$, and all $Y$ are $W$.
\risp  All $Z$ are $X$, and all $W$ are $Y$.
\risp  All $X$ are $Z$, and all $W$ are $Y$.
\risp  All $Y$ are $X$, and all $Z$ are $W$.
\risp  All $Z$ are $Y$, and all $X$ are $W$.
\risp  All $Y$ are $Z$, and all $W$ are $X$.
\end{rispm}
\end{exerm}

\begin{exerm}
  Suppose one wishes to prove that "if some $X$ are $Y$, then some $Z$ are $W$".  To do this, it would suffice to show that
\begin{rispm}
\risp[=]  All $X$ are $Z$, and all $Y$ are $W$.
\risp  Some $X$ are $Z$, and all $Y$ are $W$.
\risp  All $Z$ are $X$, and all $Y$ are $W$.
\risp  All $X$ are $Z$, and some $Y$ are $W$.
\risp  Some $Z$ are $X$, and some $Y$ are $W$.
\risp  Some $Z$ are $X$, and all $Y$ are $W$.
\risp  All $Z$ are $X$, and all $W$ are $Y$.
%% Now for the unshuffled-answer questions
\end{rispm}
\end{exerm}

\begin{exerm}
  Let $X$, $Y$, $Z$ be statements.  Suppose we know that $X$ implies $Y$, and that $Y$ implies $Z$.  If we also know that $Y$ is false, we can conclude that
%%  --ERASED--
\begin{rispm}
\risp  $X$ is false.
\risp  $Z$ is false.
\risp  $X$ implies $Z$.
\risp  $Z$ is false and $X$ implies $Z$.  Correct Answer.  $X$ is false and $X$ implies $Z$.
\risp  $X$ is false and $Z$ is false and $X$ implies $Z$.  
\risp  None of the above conclusions can be drawn.
\end{rispm}
\end{exerm}



\begin{exerm}
\begin{varianti}
%% this is an exercise with two variants (randomly select just one of the two).
\varitem 
  Let $X$, $Y$, $Z$ be statements.  Suppose we know that $X$ implies $Y$, and that $Z$ implies $X$.  If we also know that $Y$ is false, we can conclude that
\begin{rispm}
\risp  $X$ is false.
\risp  $Z$ is false.
\risp  $Z$ implies $Y$.
\risp  $Z$ is false and  $Z$ implies $Y$.
\risp  $X$ is false  and  $Z$ implies $Y$.
\risp[=]  $X$ is false, $Z$ is false, and $Z$ implies $Y$.  
\risp  None of the above conclusions can be drawn.
\end{rispm}
\varitem 
  Let $X$, $Y$,Z be statements.  Suppose we know that "$X$ is true implies $Y$ is true", and "$X$ is false implies $Z$ is true".  If we know that $Z$ is false, then we can conclude that
\begin{rispm}
\risp  $X$ is false.
\risp  $X$ is true.
\risp  $Y$ is true.
\risp[=]  $X$ is true and  $Y$ is true.
\risp  $X$ is false and $Y$ is true.
\risp  $X$ is false, $X$ is true, and $Y$ is true.  
\risp  None of the above conclusions can be drawn.
\end{rispm}

\varitem 
Let $X$, $Y$, $Z$ be statements.  Suppose we know that $X$ implies $Y$, and that $Y$ implies $Z$.  If we also know that $X$ is false, we can conclude that
\begin{rispm}
\risp  $Y$ is false.
\risp  $Z$ is false.
\risp  $Z$ implies $X$.
\risp  $Y$ is false and  $Z$ is false.
\risp  $Y$ is false and $Z$ implies $X$.
\risp  $Y$ is false, $Z$ is false and $Z$ implies $X$.  
\risp[=]  No conclusion can be drawn.
\end{rispm}
\end{varianti}
\end{exerm}


\end{esercizi*}

\end{document}

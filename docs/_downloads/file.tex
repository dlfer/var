\documentclass[a4paper]{article}
\usepackage[italian]{babel}
\usepackage{mathpazo}

%===================================================================
\usepackage[bubblesheet,sol]{mcq}
% l'opzione "bubblesheet" serve a creare la pagina con le bolle per l'OMR,
% l'opzione "sol" serve a mostrare nel pdf sia le soluzioni che i feedback
% delle soluzioni (importante: "sol" è da usare solo nella preparazione del test, non nel test vero)

\headline{Geometria I - Test 2011-07-12 (14:00-14:30, U1-08)}

\puntigiusta{3} % punti assegnati per ogni risposta giusta
\puntisbagliata{-1} % punti assegnati per ogni risposta  sbagliata
\puntiempty{0} % punti assegnati per ogni mancata risposta


\begin{document}
\bubblesheet[2]{10}{4}
% questo serve per aggiungere la scheda iniziale con le bolle da riempire,
% per 10 domande da 4 risposte l'una ABCD, poste su 2 colonne.


\begin{esercizi*}{Domande a scelta multipla}

\begin{exerm}
Se $A\subset X$ è un sottoinsieme di uno spazio topologico $X$,
allora $x\in X$ è un punto di accumulazione per
$A$ in $X$ se:
\begin{rispm}[1] 
% il parametro opzionale "1" significa che le risposte
% saranno disposte su una colonna (altrimenti il default è 2). 
\risp[=] 
% questa è la risposta giusta
In ogni intorno aperto $U$ di $x$ in $X$ ci sono
punti di $A$ diversi da $x$.
\risp In ogni intorno aperto $U$ di $x$ in $X$ ci sono punti di $A$.
\fb{Si devono avere punti diversi da $x$: in questo caso un punto
isolato non è di accumulazione, ma verifica la proprietà.}
% questo è il "feedback"; viene usato solamente per la somministrazione
% dinamica del test (moodle e html), non per il pdf. 
\risp Ogni intorno aperto $U$ di $x$ in $A$ è non-vuoto.
\fb{Se $x\in A$, allora ogni intorno aperto di $x$ in $A$
è non vuoto, ma questo non vuol dire che $x$ è di accumulazione
(potrebbe essere isolato).}
\risp Per ogni intorno $U$ di $x$ in $X$ si ha $U\cap A \neq \emptyset$.
\fb{Se $x\in A$ è un punto isolato, allora $U\cap A\neq \emptyset$,
dato che contiene $x$; ma non è di accumulazione.}
\end{rispm}
\end{exerm}

....

\end{esercizi*}
\end{document}
